\begin{tcolorbox}[enhanced jigsaw,breakable,pad at break*=1mm,
    colback=blue!5!white,colframe=babyblueeyes,title=Exercises]
    Exercises may be the most important part of this module container. We suggest that you do them actively and in small groups. Really, the only way to learn Python is to do it.      
\end{tcolorbox}
\section{Exercises 0: Getting ready}

\subsection{Installation and development environment}
Python is an open-source environment with significant input from the user community. Many of the developments are packaged in libraries designed for specific tasks. These need to be installed prior to usage which at times can be a bit tricky because of dependencies between different libraries. In order to alleviate you from installation difficulties that we have experienced in the past, we provide a fully installed python environment in a virtual environment that can be run on any laptop. If you use your own Laptop for the course this can be useful. Follow these steps to download and run the environment on your own computer:\\

\textit{XX Willi fill in details with one tiny Hello World Example in Visual Studio Code so that they know everything works. XX}\\

It is best if you do that BEFORE the first class, then we can hit the road running. If you prefer to run your own installation (let's say with Anaconda), feel free to do it: XX Required packages are matplotlib, numpy, pandas, .... XX


\ifanswers
    \begin{tcolorbox}[enhanced jigsaw,breakable,pad at break*=1mm,
    colback=blue!5!white,colframe=babyblueeyes,title=Solutions,
    watermark color=white]
    test 
    \lstinputlisting[language=Python]{Src/Ex1/HelloWorld.py}
    \end{tcolorbox}
\fi
