
\section{Data types and Visualization}
After the first subtle introduction, you should be able to fill out this table ($<$ 5 mins).
\subsection{Data Types}

    \begin{center}
      \label{tab:table1}
      \begin{tabular}{l|c|r} % <-- Alignments: 1st column left, 2nd middle and 3rd right, with vertical lines in between
        \textbf{Data Type} & \textbf{How to write in Python?} & \textbf{Geo- Env. Context}\\
        \hline
        Integer &  & \\
        Float &  & \\
        Boolean &  & \\
        List &  & \\
        Tuple &  & \\
        Dictionary &  & \\
        String &  & \\
        Numpy Array [Vector] &  & c\\
        Numpy Array [Matrix] &  & c\\
      \end{tabular}
    \end{center}
    What is the difference between an element of an array, and the index of an array?

\ifanswers
  \begin{tcolorbox}[enhanced jigsaw,breakable,pad at break*=1mm,
  colback=blue!5!white,colframe=babyblueeyes,title=Solutions,
  watermark color=white]
  \begin{center}
      \label{tab:table1}
      % \begin{tabular}{l|l|l} % <-- Alignments: 1st column left, 2nd middle and 3rd right, with vertical lines in between
        \begin{tabularx}{\linewidth}{ X | X | X }
        \textbf{Data Type} & \textbf{How to write in Python?} & \textbf{Geo- Env. Context}\\
        \hline
        Integer & \verb|a = 1| & Sample number \\
        Float & \verb|t = 20.9| & Temperature value \\
        Boolean & \verb|True, False| & Above or Below threshold? \\
        List & \verb|[20.9,| \verb|23.3,| \verb|23.1]| & Temperature time series \\
        Tuple & \verb|MLoc = ("Location",| \verb| -71.1, 8.4)| & Measurement Location \\
        Dictionary & \verb|s =| \{\verb|"Name": "samp1",| \verb|"Year": 2022|\}  & Data structure\\
        String & \verb|"Guten Tag!"| \verb|'Hello World!'| & Error message \\
        Numpy Array [Vec.] & \verb|np.array[(43.1,| \verb| 2.09, 1)]| & Time series \\
        Numpy Array [Mat.] & \verb|np.array[(43.1, 2],| \verb| [.09, 1)]| & Image \\
      \end{tabularx}
    \end{center}
  The element of an array is the value for a given index (indices are only integers). The first element of an array is evaluated with the index 0. The last one with the index -1. 
  \end{tcolorbox}
\fi


\subsection{Loading and visualization of ASCII-txt input data}
\label{sec:viskeeling}
Load the file \verb|monthly_in_situ_co2_mlo_ready4loading.txt| which is the Keeling curve. If you don't know what this is please inform yourself. The data are decimal years in the first column and CO2 in ppm in the second column. A lack of data is marked with negative numbers.
\begin{itemize}
\item Load the data (e.g., using \verb|numpy.loadtxt()|).
\item How many datapoints are in the time series? What are the dimensions of the data array?
\item Visualise the data (e.g., using \verb|matplotlib|) with  meaningful x- and y- limits and axis labels. Find out if the data are better visualised as points, curves or both. Change the color and symbols of the points.
\item Using python functions calculate basic quantities such as the mean, minimum and maximum of your time series. Are those values meaningful?
\item Visualize only the first ten elements, the last ten elements, and elements from 20 to 30.
\end{itemize}
\ifanswers
\begin{tcolorbox}[enhanced jigsaw,breakable,pad at break*=1mm,
  colback=blue!5!white,colframe=babyblueeyes,title=Solutions,
  watermark color=white]
  Test Python environment
  \lstinputlisting[language=Python]{Src/Ex1/KeelingRawData.py}
\end{tcolorbox}
\fi


\subsection{Visualization of mathematical functions}
\label{sec:mathfunc}
Make use of some numpy functions (arange) and get a feel for some vector manipulation.
\begin{itemize}
  \item Visualize $f(t)=\sin(\omega t + \phi_0)$ for a given frequency $\omega=1.2$Hz and phase shift $\phi_0$ over the time interval $t=0...10$ at discrete time intervals $dt=0.001$ s.
  \item Visualize $f^2(t)$. What happens if you reduced dt to 1 s ?
  \item Visualize a Gaussian peak:
  $$
  f(x) = Ae^{-\frac{(x-x_0)^2}{\sigma_x}}
  $$
  where $x$ is the independent variable and the factors $x_0$, $A$, and $\sigma_x$ determine the shape of the function. Make sure that you understand what each parameter pertains to.
\end{itemize}

\subsection{Basic filtering and data manipulation}
\label{sec:loops}
Load the Keeling curve from \ref{sec:viskeeling} again. We have understood that no data values are marked with negative numbers. Let's remove those with a for loop (there are better ways at a later stage). Take your time, if this is your first time of writing a loop this can take some time.
\begin{itemize}
  \item Print all CO$_2$ values on the screen using a for loop. (How could this be done much easier?)
  \item Print only the time intervals where no data are available
  \item Create a new vector where the no-data values are removed.
  \item Visualize the new vector. Anything different compared to \ref{sec:viskeeling}?
\end{itemize}
Start a new for loop block and calculate the time derivative of the time series. Visualize it. 
\ifanswers
\begin{tcolorbox}[enhanced jigsaw,breakable,pad at break*=1mm,
  colback=blue!5!white,colframe=babyblueeyes,title=Solutions,
  watermark color=white]
  Test Python environment
  \lstinputlisting[language=Python]{Src/Ex1/KeelingFilter.py}
\end{tcolorbox}
\fi
\subsection{Loops continued.}
Visualize a 2D Gaussian peak:
$$
f(x,y) = Ae^{-\frac{(x-x_0)^2}{\sigma_x}-\frac{(y-y_0)^2}{\sigma_y}}
$$
using a nested for loop. This is harder. In essence you will need to fill a 2D array and visualize it with colors using \verb|plt.pcolormesh()|. Then do the same but replace the nested for loops using numpys \verb|"meshgrid"|. What is easier?
