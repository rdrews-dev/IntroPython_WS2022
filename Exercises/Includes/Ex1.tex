\begin{tcolorbox}[enhanced jigsaw,breakable,pad at break*=1mm,
    colback=blue!5!white,colframe=babyblueeyes,title=Informal Presentations]
    In this exercise you will develop presentations for your fellow students. The presentations will be informal. This means that you don't practice the presentation, you don't spend time on making slides nice, you don't worry about timing and it's ok to have some knowledge gaps. The presentation's only intention is to inform your fellow students about some of your findings and to stirr discussions. Only choose one presentation out of 2.1 and 2.2.2.      
\end{tcolorbox}
\section{Introduction}

\subsection{Electrical properties in petroleum exploration}
A classical paper in the petroleum industry was written by Gus Archie in 1942. Summarize some main findings of this paper and compare it section 2.2.4.1 in Binley \& Slater. Prepare an approximately 10-15 minute informal presentation for your peers.

\subsection{Literature survey of applied resistivity mapping}

There are a number of applications for resistivity mapping. Find examples for:
\begin{itemize}
    \item Groundwater studies
    \item Contaminant plumes
    \item Soil-rock interfaces
    \item Caves
    \item Waste-disposal sites
    \item Mineral exploration
    \item others
\end{itemize}
Prepare an approximately 10-15 informal presentation with case studies for your peers. Summarize the source, a core image, and a few sentences for each example on a joint PDF merged from all groups.


\ifanswers
    \begin{tcolorbox}[enhanced jigsaw,breakable,pad at break*=1mm,
    colback=blue!5!white,colframe=babyblueeyes,title=Solutions,
    watermark color=white]
    An example summary is here: PapersForStudents/ApplicationsOfResistivity\_223B7\-2009.pdf
    
   % \lstinputlisting[language=Python]{source_filename.py}
    \end{tcolorbox}
\fi