\section{More loops}

\subsection{Endless loops, reading user input}
Almost all programming languages that support loops can also have endless (infinite) loops, that is loops that never end.
Besides a \verb|for| loop Python also has a \verb|while| loop. The syntax is very easy: \\


\begin{tcolorbox}[enhanced jigsaw,breakable,pad at break*=1mm,
    colback=blue!5!white,colframe=babyblueeyes,title=While loop 1,
    watermark color=white]
    \lstinputlisting[language=Python]{Src/Ex3/While1.py}
\end{tcolorbox}

To finish a loop in Python just use the \verb|break| keyword: \\

\begin{tcolorbox}[enhanced jigsaw,breakable,pad at break*=1mm,
    colback=blue!5!white,colframe=babyblueeyes,title=While loop 2,
    watermark color=white]
    \lstinputlisting[language=Python]{Src/Ex3/While2.py}
\end{tcolorbox}

In order to read an input from the user via keyboard Python has the \verb|input()| function: \\

\begin{tcolorbox}[enhanced jigsaw,breakable,pad at break*=1mm,
    colback=blue!5!white,colframe=babyblueeyes,title=User input 1,
    watermark color=white]
    \lstinputlisting[language=Python]{Src/Ex3/Input1.py}
\end{tcolorbox}

To convert a string into a number Python has two functions: \verb|int()| and \verb|float()|. \\

Now write a function that reads in numbers from the user and calculates the total sum and prints this value at each iteration.
The program should also print out all the numbers sorted (ascending) at the end. In order to quit the program the user has to enter \verb|'quit'| instead of a number. \\

\ifanswers
\begin{tcolorbox}[enhanced jigsaw,breakable,pad at break*=1mm,
    colback=blue!5!white,colframe=babyblueeyes,title=Solutions,
    watermark color=white]
    User input 2
    \lstinputlisting[language=Python]{Src/Ex3/UserInput.py}
\end{tcolorbox}
\fi

\subsection{Recursion}

There is a third way to do loops in Python (and other programming languages as well):
If you define a function you can call that function recursively: \\

\begin{tcolorbox}[enhanced jigsaw,breakable,pad at break*=1mm,
    colback=blue!5!white,colframe=babyblueeyes,title=Recursion with Fibonacci,
    watermark color=white]
    \lstinputlisting[language=Python]{Src/Ex3/Fibonacci1.py}
\end{tcolorbox}

Mathematically the Fibonacci numbers are defined as: \\

\begin{math}
    F_{n} = \left\{
        \begin{array}{ll}
            0, & n = 0 \\
            1, & n = 1 \\
            F_{n-1} + F_{n-2}, & n > 1
        \end{array}
    \right.
\end{math}

Now write the Fibonacci function in a non-recursive way using a \verb|for| loop or a \verb|while| loop. \\

\ifanswers
\begin{tcolorbox}[enhanced jigsaw,breakable,pad at break*=1mm,
    colback=blue!5!white,colframe=babyblueeyes,title=Solutions,
    watermark color=white]
    \lstinputlisting[language=Python]{Src/Ex3/Fibonacci2.py}
\end{tcolorbox}
\fi

\subsection{Reading data from multiple files}
At some time you have to read in multiple files one after each other and combine all the data.
One way of doing this is to specify the files in a list: \\

\begin{tcolorbox}[enhanced jigsaw,breakable,pad at break*=1mm,
    colback=blue!5!white,colframe=babyblueeyes,title=Multiple files 1,
    watermark color=white]
    \lstinputlisting[language=Python]{Src/Ex3/Files1.py}
\end{tcolorbox}

Another way is to specify a file pattern and to load all the files that have that pattern in their name: \\

\begin{tcolorbox}[enhanced jigsaw,breakable,pad at break*=1mm,
    colback=blue!5!white,colframe=babyblueeyes,title=Multiple files 2,
    watermark color=white]
    \lstinputlisting[language=Python]{Src/Ex3/Files2.py}
\end{tcolorbox}

So now try to do it by yourself: Load in some data files, merge the data and make a nice plot.
